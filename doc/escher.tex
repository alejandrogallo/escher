\documentclass[a4paper]{article}
% generated by Docutils <http://docutils.sourceforge.net/>
\usepackage{fixltx2e} % LaTeX patches, \textsubscript
\usepackage{cmap} % fix search and cut-and-paste in Acrobat
\usepackage{ifthen}
\usepackage[T1]{fontenc}
\usepackage[utf8]{inputenc}
\usepackage{color}
\usepackage[pdftex]{graphicx}

\setcounter{secnumdepth}{0}
\usepackage{textcomp} % text symbol macros

%%% Custom LaTeX preamble
% PDF Standard Fonts
\usepackage{mathptmx} % Times
\usepackage[scaled=.90]{helvet}
\usepackage{courier}

%%% User specified packages and stylesheets
%\usepackage[latin1]{inputenc}
%\usepackage[T1]{fontenc}
%\usepackage{lmodern}
\usepackage{ucs}
\usepackage{type1cm}
\ifthenelse{\isundefined{\definecolor}}{
  \usepackage[pdftex]{color}
}{}
\usepackage{pdfcolmk} % Solve problems with the color stack
\usepackage{listings}
\usepackage{url}
\usepackage{amsmath}
\usepackage[pdftex,margin=1.8cm,twosideshift=0pt,verbose=true]{geometry}

\usepackage{sectsty}
\allsectionsfont{\sffamily}
\setcounter{secnumdepth}{4}
\setcounter{tocdepth}{4}

\setlength{\parindent}{0pt}
\setlength{\parskip}{6pt plus 2pt minus 1pt}

\definecolor{White}{rgb}{1,1,1}
\definecolor{Black}{rgb}{0,0,0}
\definecolor{Red}{rgb}{1,0,0}                      % Text
\definecolor{Green}{rgb}{0,1,0}
\definecolor{Blue}{rgb}{0,0,1}
\definecolor{SaddleBrown}{rgb}{0.55,0.27,0.07}     % Page borders
\definecolor{Blue2}{rgb}{0,0,0.8}
\definecolor{Gold}{rgb}{1,0.84,0}
\definecolor{MintCream}{rgb}{0.96,1,0.98}          % Page background
\definecolor{NavyBlue}{rgb}{0,0,0.5}               % Title page
\definecolor{Green3}{rgb}{0,0.5,0}
\definecolor{DarkSlateGray}{rgb}{0.18,0.31,0.31}
\definecolor{Purple2}{rgb}{0.57,0.17,0.93}
\definecolor{Copper}{rgb}{0.90,0.35,0.15}
\definecolor{Brown}{rgb}{0.647,0.165,0.165}
\definecolor{Aquamarine}{rgb}{0.439216,0.858824,0.576471}
\definecolor{DarkTurquoise}{rgb}{0.439216,0.576471,0.858824}
\definecolor{Firebrick}{rgb}{0.556863,0.137255,0.137255}
\definecolor{DarkCyan}{rgb}{0,0.5,0.5}
\definecolor{MandarinOrange}{rgb}{0.89,0.47,0.20}
\definecolor{Gray}{rgb}{0.36,0.51,0.51}
\definecolor{LightGray}{rgb}{0.85,0.90,0.90}
\definecolor{DarkGreen}{rgb}{0.063,0.427,0.176}
\definecolor{DarkPurple}{rgb}{0.447,0.086,0.424}
\definecolor{SeaGreen}{rgb}{0.137255,0.556863,0.419608}
\definecolor{WhiteBg}{rgb}{1,1,1}
\definecolor{GoldBg}{rgb}{1,0.84,0}
\definecolor{MintBg}{rgb}{0.76,1,0.90}
\definecolor{NavyBg}{rgb}{0,0,0.5}
\definecolor{YlBg}{rgb}{1,0.84,0.5}
\definecolor{OrangeBg}{rgb}{1.00,0.70,0.40}
\definecolor{PaleBlueBg}{rgb}{0.706,0.871,1}
\definecolor{PaleRoseBg}{rgb}{1,0.765,0.812}
\definecolor{GreyBg}{rgb}{0.85,0.85,0.85}
\definecolor{PaleYlBg}{rgb}{1,0.980,0.700}
\definecolor{Gray85Bg}{rgb}{0.90,0.90,0.90}
%\definecolor{OctaveBg}{rgb}{0.902,0.745,0.745}
\definecolor{OctaveBg}{rgb}{1.000,0.827,0.620}

\ifthenelse{\isundefined{\hypersetup}}{
  \usepackage[colorlinks=true,linkcolor=DarkGreen,urlcolor=DarkGreen]{hyperref}
}{
  \hypersetup{colorlinks=true,linkcolor=DarkGreen,urlcolor=DarkGreen}
}

\lstset{%
     inputencoding=utf8x,%
     extendedchars=true,%
     xleftmargin=0em,%
     showstringspaces=false,%
     backgroundcolor=\color{PaleYlBg},%
     fillcolor=\color{PaleYlBg},%
     frame=single,framerule=0pt,%
     keywordstyle=\color{Brown}\bfseries,%
     commentstyle=\color{Blue},%
     numbers=left,%
     stepnumber=1,%
     numberstyle=\tiny,%
     xleftmargin=1em,%
     lineskip=-1pt,%
     language=critic,%
     basicstyle=\ttfamily}

\newcommand{\asciilist}{
\lstset{%
     inputencoding=utf8x,%
     extendedchars=true,%
     xleftmargin=0em,%
     showstringspaces=false,%
     backgroundcolor=\color{MintBg},%
     fillcolor=\color{MintBg},%
     frame=single,framerule=0pt,%
     keywordstyle=\color{Brown}\bfseries,%
     commentstyle=\color{Blue},%
     numbers=left,%
     stepnumber=1,%
     numberstyle=\tiny,%
     xleftmargin=1em,%
     lineskip=-1pt,%
     language=,%
     basicstyle=\ttfamily}
}

\newcommand{\criticlist}{
\lstset{%
     inputencoding=utf8x,%
     extendedchars=true,%
     xleftmargin=0em,%
     showstringspaces=false,%
     backgroundcolor=\color{PaleYlBg},%
     fillcolor=\color{PaleYlBg},%
     frame=single,framerule=0pt,%
     keywordstyle=\color{Brown}\bfseries,%
     commentstyle=\color{Blue},%
     numbers=left,%
     stepnumber=1,%
     numberstyle=\tiny,%
     xleftmargin=1em,%
     lineskip=-1pt,%
     language=critic,%
     basicstyle=\ttfamily}
}

\newcommand{\runwienlist}{
\lstset{%
     inputencoding=utf8x,%
     extendedchars=true,%
     xleftmargin=0em,%
     showstringspaces=false,%
     backgroundcolor=\color{PaleYlBg},%
     fillcolor=\color{PaleYlBg},%
     frame=single,framerule=0pt,%
     keywordstyle=\color{Brown}\bfseries,%
     commentstyle=\color{Blue},%
     numbers=left,%
     stepnumber=1,%
     numberstyle=\tiny,%
     xleftmargin=1em,%
     lineskip=-1pt,%
     language=runwien,%
     basicstyle=\ttfamily}
}

\newcommand{\octavelist}{
\lstset{%
     inputencoding=utf8x,%
     extendedchars=true,%
     xleftmargin=0em,%
     showstringspaces=false,%
     backgroundcolor=\color{OctaveBg},%
     fillcolor=\color{OctaveBg},%
     frame=single,framerule=0pt,%
     keywordstyle=\color{Brown}\bfseries,%
     commentstyle=\color{Blue},%
     numbers=left,%
     stepnumber=1,%
     numberstyle=\tiny,%
     xleftmargin=1em,%
     lineskip=-1pt,%
     language=Octave,%
     basicstyle=\ttfamily}
}

\newcommand{\tessellist}{
\lstset{%
     inputencoding=utf8x,%
     extendedchars=true,%
     xleftmargin=0em,%
     showstringspaces=false,%
     backgroundcolor=\color{PaleYlBg},%
     fillcolor=\color{PaleYlBg},%
     frame=single,framerule=0pt,%
     keywordstyle=\color{Brown}\bfseries,%
     commentstyle=\color{Blue},%
     numbers=left,%
     stepnumber=1,%
     numberstyle=\tiny,%
     xleftmargin=1em,%
     lineskip=-1pt,%
     language=tessel,%
     basicstyle=\ttfamily}
}

\newcommand{\gibbslist}{
\lstset{%
     inputencoding=utf8x,%
     extendedchars=true,%
     xleftmargin=0em,%
     showstringspaces=false,%
     backgroundcolor=\color{PaleYlBg},%
     fillcolor=\color{PaleYlBg},%
     frame=single,framerule=0pt,%
     keywordstyle=\color{Brown}\bfseries,%
     commentstyle=\color{Blue},%
     numbers=left,%
     stepnumber=1,%
     numberstyle=\tiny,%
     xleftmargin=1em,%
     lineskip=-1pt,%
     language=gibbs2,%
     basicstyle=\ttfamily}
}


%%% Fallback definitions for Docutils-specific commands

% fieldlist environment
\ifthenelse{\isundefined{\DUfieldlist}}{
  \newenvironment{DUfieldlist}%
    {\quote\description}
    {\enddescription\endquote}
}{}

% hyperlinks:
\ifthenelse{\isundefined{\hypersetup}}{
  \usepackage[colorlinks=true,linkcolor=blue,urlcolor=blue]{hyperref}
  \urlstyle{same} % normal text font (alternatives: tt, rm, sf)
}{}


%%% Body
\begin{document}

!!THIS DOCUMENT IS OUTDATED!!
DEVS.ORG IS THE ONLY DOCUMENT THAT IS KEPT UP TO DATE FOR NOW.
The installation section may apply, though.


\section{1~~~MolWare user's and reference guide%
  \label{molware-user-s-and-reference-guide}%
}
%
\begin{DUfieldlist}
\item[{Author:}]
Víctor Luaña (VLC) and Alberto Otero-de-la-Roza (AOR)

\item[{Contact:}]
\href{mailto:victor@carbono.quimica.uniovi.es}{victor@carbono.quimica.uniovi.es}

\item[{Address:}]
Departamento de Química Física y Analítica, Universidad de Oviedo,
Principado de Asturias,
Julián Clavería 8, 33007 Oviedo, Spain

\item[{Contact:}]
\href{mailto:aoterodelaroza@ucmerced.edu}{aoterodelaroza@ucmerced.edu}

\item[{Address:}]
School of Natural Sciences,
University of California, Merced, 5200 North Lake Road, Merced,
California 95343, USA.

\item[{Version:}]
0.1 (2011-12-23)

\end{DUfieldlist}

\thispagestyle{empty}
\enlargethispage{+1\baselineskip}

\noindent\makebox[\textwidth][c]{\includegraphics[scale=0.650000]{molware.png}}

\clearpage

\phantomsection\label{contents}
\pdfbookmark[2]{Contents}{contents}
\tableofcontents


\clearpage


\subsection{1.1~~~Introduction%
  \label{introduction}%
}

MolWare is a collection of octave routines and awk scripts developed with
the purpose of working with molecules and crystals and interfacing with
other codes that work with molecules and crystals.

The package includes a number of scripts designed to analyze the output of
several electronic structure codes and extract relevant information from
there:
%
\begin{quote}
%
\begin{itemize}

\item extract-bond.awk:

\item extract-del2rho.awk:

\item extract-extreme.awk:

\item extract-g98.awk:

\item extract-gamess.awk:

\item extract-gmseorb.awk:

\item extract-irc.awk:

\item extract-LCP.awk:

\item extract-pn.awk:

\item extract-proaimv.awk:

\item extract\_cryst\_qe.awk - Analysis of the quantum espresso output that
corresponds to the optimization of a molecular crystal. The last geometry
is converted to a tessel input.

\item extract\_txyz\_promolden.awk: extracts xyz coordinates of the atoms and
the critical points, plust the scalar values of the cps, from a
promolden output of the topology of an scalar field.

\item extract\_txyz\_critic2g.awk: extracts xyz coordinates of the atoms and
the critical points, from a critic2g output of the topology of an scalar
field.

\item extract\_xyz\_nwchem.awk: analyzes the output of a NWChem calculation
and extracts xyz files corresponding to the different geometries
found. See the ``NWChem notes'' section.

\item extract\_xyz\_g09.awk:

\item extract\_xyz\_gamess.awk:

\end{itemize}

\end{quote}

There are also a collection of octave based complete tasks:
%
\begin{quote}
%
\begin{itemize}

\item get\_geom\_xyz.m: get the geometry of an xyz file.

\item get\_xyz\_geom\_c6x6\_yz3.m: geometry of an c6x6\_yz3 adduct.

\end{itemize}

\end{quote}

Finally, the standalone octave programs are made of a collection of
routines:
%
\begin{quote}
%
\begin{itemize}

\item cr\_read\_espresso.m: read in the optimized geometry for a crystal from
a quantum espresso (pwscf) calculation.

\item cr\_read\_vasp.m: read in the crystal geometry from a vasp calculation
(POSCAR and POTCAR).

\item cr\_write\_vasp.m: write a vasp-style crystal geometry (POSCAR).

\item cr\_write\_cif.m: write the crystal description as a cif file.

\item cr\_molmotif.m: extracts one or more molecular motifs from a molecular
crystal description. Similar to the molmotif routine in tessel.

\item cr\_crystalbox.m: extract a molecular description from a
parallelpiped in crystallographic coordinates.

\item cr\_qewald.m: calculate the electrostatic energy of a lattice of
point charges using the Ewald method.

\item cr\_vdwewald.m: calculate the dispesrion energy of a crystal
(R\textasciicircum{}\{-6\} lattice sum) using Ewald method.

\item cr\_popinfo.m: write some information about a crystal struct to
a file or to standard output.

\item cr\_xrd.m: calculates the x-ray powder diffractogram.

\item cr\_dbxrd.m: atomic scattering factor database.

\item cr\_xrd\_gnuplot.m: uses the output from cr\_xrd.m to plot the
powder diffraction diagram.

\item color.m: transform a X11 color name into a rgb triplet.

\item cylindermodel.m: create the vertices and faces of a normalized cylinder.

\item mol\_2molsgeometry.m: geometry between two different molecules.

\item mol\_addatom.m: add a new atom to a molecular database.

\item mol\_adduct.m: forms a new molecule from two fragments.

\item mol\_align.m: least square align a part of a molecule to the equivalent
part of another molecule.

\item mol\_angle.m: returns the angle between three atoms in the molecule.

\item mol\_ball.m: create a ball representation from a molecule.

\item mol\_classify.m: classify atoms in a molecule by atomic number and binding.

\item mol\_cmass.m: determines the center of mass of a molecule.

\item mol\_cuberange.m: prapare the input of cubegen for a molecule.

\item mol\_dbatom.m: get the properties of an atom from its atomic symbol.

\item mol\_dbstart.m: initialize the atomic properties database.

\item mol\_dbsymbol.m: get the properties of an atom from its atomic number.

\item mol\_dist.m: returns the distance between two atoms in the molecule.

\item mol\_dist2.m: returns the matrix of distances between groups of atoms
in two molecules.

\item mol\_fsck2topo.m: automatic MEP analysis from %
\raisebox{1em}{\hypertarget{id2}{}}\hyperlink{id1}{\textbf{\color{red}*}}.fsck.

\item mol\_getfragment.m: copies part of a molecule into a new fragment.

\item mol\_groupgeometry.m: geometry between two different fragments in
a molecule.

\item mol\_inertiamatrix.m: determines and diagonalizes the inertia matrix.

\item mol\_internalgeometry.m: obtains the connectivity between atoms and
the non connected fragments in a molecule.

\item mol\_islinear.m: checks if the molecule is linear.

\item mol\_isnew.m: checks if some coordinates are already in the molecule.

\item mol\_isplanar.m: checks if the molecule is planar.

\item mol\_limits.m: gets a bounding box for the molecule:

\item mol\_readcube.m: reads a gaussian cube file.

\item mol\_read\_fchk.m: reads the molecular geometry from a gaussian fchk file.

\item mol\_readxyz.m: reads a molecule from a xyz file.

\item mol\_reorder.m: reorders the list of atoms.

\item mol\_smiles2xyz.m: uses smiles (and open babel) to produce the xyz
coordinates of a molecule.

\item mol\_stick.m: create a stick representation from a molecule.

\item mol\_transform.m: applies a 3x4 (rotation+traslation) matrix.

\item mol\_uniqatoms.m: detects atoms closer than eps and returns a list of
the unique atoms.

\item mol\_unitconvert.m: converts units in a mol description.

\item mol\_writeg09.m: prepares a gaussian09 input file for a molecule.

\item mol\_writenw.m: prepares a NWChem input file.

\item mol\_writexyz.m: writes a molecule to a xyz file.

\item op\_prod.m: product of two rotation-translation operations (Seitz op.)
Each operation is 3x4 and it is assumed to work on a 3x1 column
vector.

\item op\_rot3D.m: returns the matrix corresponding to a rotation in 3D,
defined in terms of three Euler angles, plus a translation.

\item op\_rotx.m: returns the matrix corresponding to a counter clockwise
rotation of ``angle'' degrees around the x axis plus the addition
translation.

\item op\_roty.m and op\_rotz.m:

\item zmat\_step.m: calculates the coordinates of new atom given three points
and a zmat reference: distance, angle, and dihedral angle.

\item rep\_addcamera.m: add a camera to a scene.

\item rep\_addlight.m: add a light to a scene.

\item rep\_setdefaultscene.m: add camera, lights and background to a
graphical representation using the objects in it and reasonable
default values.

\item rep\_write\_obj.m: write a wavefront obj file from a graphical
representation.

\item rep\_write\_off.m: write a geomview (OFF) file from a graphical
representation.

\item rep\_write\_pov.m: write a povray file from a graphical
representation.

\item rep\_texdbstart.m: initialize the texture database.

\item rep\_texture.m: retrieve a texture from the database.

\item rep\_addpovtexture.m: add a pov texture to the database.

\item rep\_addobjtexture.m: add an obj texture to the database.

\item rep\_setbgcolor.m: change the background color of a scene.

\item spheremodel.m: create the vertices and faces of a normalized sphere.

\end{itemize}

\end{quote}


\subsubsection{The molecular structure%
  \label{the-molecular-structure}%
}


\subsubsection{The crystal structure%
  \label{the-crystal-structure}%
}

The structure describing a crystal (cr) contains the following fields:
%
\begin{itemize}

\item cr.name : name of the crystal, only for labelling purposes.

\item cr.ntyp : number of atomic species (types).

\item cr.zvaltyp(1,cr.ntyp) : valence atomic number for each atomic
type.

\item cr.attyp\{1,cr.ntyp\} : atomic symbols for each atomic type.

\item cr.ztyp(1,cr.ntyp) : atomic numbers for each atomic type.

\item cr.qtyp(1,cr.ntyp) : atomic formalx charges. Used by qewald.m in the
calculation of the electrostatic lattice energy. Also, by xrd.m to
load the correct atomic scattering factors from the database.

\item cr.typcount(1,cr.ntyp) : number of atoms of each type.

\item cr.c6typ(cr.ntyp,cr.ntyp) : matrix of dispersion interaction
coefficients. Used in cr\_vdwewald.m.

\item cr.rvdwtyp(cr.ntyp,cr.ntyp) : matrix of van der Waals radii. Used
in cr\_vdwewald.m only if damping is activated.

\item cr.nat : total number of atoms in the unit cell.

\item cr.typ(1,cr.nat) : type of each atom in the unit cell. cr.typ(i)
is index to the cr.*typ arrays.

\item cr.x(cr.nat,3) : crystallographic coordinates for each atom in the
unit cell.

\item cr.r(3,3) : crystallographic to cartesian matrix (bohr).

\item cr.g(3,3) : metric tensor (bohr\textasciicircum{}2).

\item cr.a(1,3) : cell lengths (bohr).

\item cr.b(1,3) : cell angles (radians).

\item cr.omega : cell volume (bohr\textasciicircum{}3).

\end{itemize}


\subsubsection{The graphical representations%
  \label{the-graphical-representations}%
}
%
\begin{itemize}

\item rep.name: title of the representation.

\item rep.nball: number of balls.

\item rep.ball\{1:rep.nball\}: cell array of balls. Each ball is a
structure, with the following fields:
- rep.ball\{i\}.name: (atomic) name associated to the ball.
- rep.ball\{i\}.x: coordinates of the ball in angstrom.
- rep.ball\{i\}.r: radius (angstrom).
- rep.ball\{i\}.rgb: color triplet (from 0 to 255).

\item rep.nstick: number of sticks.

\item rep.stick\{1:rep.stick\}: cell array of sticks. Each stick is a
structure, with the following fields:
- rep.stick\{i\}.name: name associated to the stick.
- rep.stick\{i\}.x0: coordinates of the first point.
- rep.stick\{i\}.x1: coordinates of the second point.
- rep.stick\{i\}.r: radius.
- rep.stick\{i\}.rgb: color triplet (from 0 to 255).

\item rep.cam: a structure containing the definition of the camera. Its
fields are (all units are angstrom):
- rep.cam.cop: the position of the camera.
- rep.cam.sky: the sky direction.
- rep.cam.vuv: the up direction.
- rep.cam.rht: the right direction.
- rep.cam.drt: the third direction.
- rep.cam.vrp: the point the camera is pointing at.

\item rep.nlight: the number of lights

\item rep.light\{1:rep.nlight\}: a cell array with the descriptions of the
lights. It is:
- rep.light\{i\}.x: coordinates of the light (angstrom).
- rep.light\{i\}.color: color string.

\item rep.bgcolor: the background color as an integer triplet (0 to 255).

\end{itemize}


\subsubsection{Installation of the package on a unix-like operating system%
  \label{installation-of-the-package-on-a-unix-like-operating-system}%
}

The MolWare package is distributed as a single compressed tar file. Let
us assume that you want to install it in one of your personal
directories, for instance in ''\textasciitilde{}/src/''. Then you should
%
\asciilist
\begin{lstlisting}
mv molware.tgz ~/src
cd ~/src
tar xtvf molware.tgz
\end{lstlisting}

This will create the tree of directories
%
\asciilist
\begin{lstlisting}
~/src/molware/doc/
~/src/molware/src/
~/src/molware/test/
\end{lstlisting}

\texttt{doc} contains this documentation, \texttt{src} is the home of all the
routines forming the package, and \texttt{test} contains a set of scripts and
data files for testing. The \texttt{src} directory must be added to the path
where octave looks for files and routines. This can be done adding the
next line to the \texttt{\textasciitilde{}/.octaverc} configuration file:
%
\asciilist
\begin{lstlisting}
addpath("~/src/molware/src/");
\end{lstlisting}

Finally, we have found very useful to create a symbolic link of the main
tasks in the binary directory where most of the personal executables can
be found
%
\asciilist
\begin{lstlisting}
cd ~/bin
ln -s ~/src/molware/src/extract_xyz_nwchem.awk extract_xyz_nwchem.awk
\end{lstlisting}

In this way, just by making sure that \texttt{\textasciitilde{}/bin} is included in the path
of binaries, we can use asturfit and the rest of tasks on any working
directory.

The package can also be installed in a system directory for the access
of all the users and added to the general octave path.


\paragraph{Instalation of the \emph{escherpy} module in Fedora 19%
  \label{instalation-of-the-escherpy-module-in-fedora-19}%
}

Installation instructions to run the 'escherpy' module.
More or less the same with other distributions.
%
\begin{itemize}

\item Install gcc make automake

\end{itemize}
%
\asciilist
\begin{lstlisting}
sudo yum install make automake gcc gcc-c++
\end{lstlisting}
%
\begin{itemize}

\item Install cmake-2.8.12.1 or >=2.8

\end{itemize}
%
\asciilist
\begin{lstlisting}
sudo yum install cmake
#./bootstrap
#  make
\end{lstlisting}
%
\begin{itemize}

\item Install git

\end{itemize}
%
\asciilist
\begin{lstlisting}
sudo yum install git
\end{lstlisting}
%
\begin{itemize}

\item Download VTK

\end{itemize}
%
\asciilist
\begin{lstlisting}
git clone git://vtk.org/VTK.git
cd VTK
I'm sure release candidate 3 works. I cannot assure others
will work. It must be v6 at least.
git checkout v6.0.0.rc3
\end{lstlisting}
%
\begin{itemize}

\item Install dependencies of VTK

\end{itemize}
%
\asciilist
\begin{lstlisting}
sudo yum install freeglut freeglut-devel
sudo yum install libXt-devel
sudo yum install python-devel
\end{lstlisting}
%
\begin{itemize}

\item Install VTK

\end{itemize}
%
\asciilist
\begin{lstlisting}
ccmake .
Type c to configure
Turn ON BUILD_SHARED_LIBS
It is advisable that you change
CMAKE_INSTALL_PREFIX to a folder in your $HOME
Turn ON VTK_WRAP_PYTHON
Type c to configure
Again to be sure:
Type c to configure
Type g to generate configuration file
make
make install
echo 'export PYTHONPATH=$PYTHONPATH:/path/to/VTK/bin/' >> $HOME/.bashrc
echo 'export PYTHONPATH=$PYTHONPATH:/path/to/VTK/lib/' >> $HOME/.bashrc
echo 'export PYTHONPATH=$PYTHONPATH:/path/to/VTK/Wrapping/Python/' >> $HOME/.bashrc
echo 'export LD_LIBARY_PATH=$LD_LIBARY_PATH:/path/to/prefix/lib/' >> $HOME/.bashrc
\end{lstlisting}

check if it works:
%
\asciilist
\begin{lstlisting}
python -c 'import vtk'
\end{lstlisting}
%
\begin{itemize}

\item Install numpy

\end{itemize}
%
\asciilist
\begin{lstlisting}
sudo yum install numpy
\end{lstlisting}

Define:
%
\asciilist
\begin{lstlisting}
echo 'export ESCHER_DATA=/path/to/escher_data/' >> $HOME/.bashrc
echo 'export PYTHONPATH=$PYTHONPATH:/path/to/escher/' >> $HOME/.bashrc
\end{lstlisting}

Optionally, you can let the log write detailed info about the run,
otherwise set it to 'OFF':
%
\asciilist
\begin{lstlisting}
echo 'export ESCHER_DEBUG="ON"' >> $HOME/.bashrc
\end{lstlisting}

Run
%
\asciilist
\begin{lstlisting}
source ~/.bashrc
\end{lstlisting}

to refresh your environment variables.

The \emph{parse} python module is planned to be used in the future:
%
\asciilist
\begin{lstlisting}
sudo pip install parse
\end{lstlisting}

or
%
\asciilist
\begin{lstlisting}
sudo easy_install parse
\end{lstlisting}

Go to \$ESCHER\_HOME/test and run any testme.py:
%
\asciilist
\begin{lstlisting}
./testme.py
\end{lstlisting}

If the following error occurs
%
\asciilist
\begin{lstlisting}
Traceback (most recent call last):

File "./testme.py", line 5, in <module>
    import escherpy as esc
ImportError: No module named escherpy
\end{lstlisting}

the escher path has not been added properly to your PYTHONPATH.

After that, several 3D files will appear and a Blender script
to visualize the VRML file. It can be run with:
%
\asciilist
\begin{lstlisting}
blender -P vrml.bpy
\end{lstlisting}


\paragraph{Use of escherpy%
  \label{use-of-escherpy}%
}

Complete set of instructions.
%
\asciilist
\begin{lstlisting}
import escherpy as esc

mol = esc.Molecule()

mol. structfile = 'file path'
mol.readstruct()
mol.rot(60., 30., 20.)
mol.stickball()

mol. cpfile = 'file path'
mol.readcps()
mol.cpball()

mol.basinfile = 'file path'
mol.readsurf('Na')

mol.isovalue = 0.1
mol.isosurface('file path')

mol. densfile = 'file path'
mol. gradfile = 'file path'
mol.nciplot()

mol.show()
\end{lstlisting}


\subsubsection{Citation of this package%
  \label{citation-of-this-package}%
}

Please, consider citing ref. \cite{mw1} if you find the MolWare package
useful for your work.


\subsubsection{Compatibility with MatLab%
  \label{compatibility-with-matlab}%
}

MolWare is made of octave routines and scripts. No attempt has been made
to ensure compatibility with MatLab. However, we anticipate only a few
potential problems:
%
\begin{quote}
\newcounter{listcnt0}
\begin{list}{(\arabic{listcnt0})}
{
\usecounter{listcnt0}
\setlength{\rightmargin}{\leftmargin}
}

\item the ''endfunction'', ''endfor'', and ''endif'' should be converted
to simple ''end''s;
\end{list}

\end{quote}


\subsection{1.2~~~Some examples%
  \label{some-examples}%
}


\subsubsection{Test01: Reading a xyz file and analyzing the molecule%
  \label{test01-reading-a-xyz-file-and-analyzing-the-molecule}%
}


\subsection{1.3~~~Notes concerning the electronic structure codes%
  \label{notes-concerning-the-electronic-structure-codes}%
}


\subsubsection{g03%
  \label{g03}%
}


\subsubsection{g09%
  \label{g09}%
}


\subsubsection{gamess%
  \label{gamess}%
}


\subsubsection{NWchem%
  \label{nwchem}%
}


\subsection{1.4~~~Format of relevant I/O files%
  \label{format-of-relevant-i-o-files}%
}


\subsubsection{xyz files with coordinates%
  \label{xyz-files-with-coordinates}%
}


\subsection{1.5~~~Internal representation of a molecule%
  \label{internal-representation-of-a-molecule}%
}


\subsection{1.6~~~Alphabetic list of routines and routine documentation%
  \label{alphabetic-list-of-routines-and-routine-documentation}%
}


\subsubsection{mol\_addatom.m%
  \label{mol-addatom-m}%
}
%
\octavelist
\begin{lstlisting}
function molout = mol_addatom (atname, atxyz, molin, newmol=0, LOG=1)
\end{lstlisting}

Add a new atom (if it is not included) to the molecule.

Required input variables:
%
\begin{itemize}

\item atmane: name of new atom.

\item atxyz: (1:3) array with the cartesian coordinates of the new atom.

\item %
\begin{description}
\item[{molin: structure with the input molecular description. The format is:}] \leavevmode %
\begin{itemize}

\item molin.name -{}-{}-> title of the molecule.

\item %
\begin{description}
\item[{molin.atname -{}-> \{1:M\} cell array with the symbols of the atoms}] \leavevmode 
(M is the number of atoms in the molecule).

\end{description}

\item molin.xyz -{}-{}-{}-{}-> Mx3 matrix with the atomic coordinates.

\item molin.mass -{}-{}-{}-> {[}1:M{]} vectos with atomic masses.

\end{itemize}

\end{description}

\end{itemize}

Optional input variables (all have default values):
%
\begin{itemize}

\item %
\begin{description}
\item[{\{LOG = 1\}: print information about the data read in if LOG>0.}] \leavevmode %
\begin{itemize}

\item LOG = 0  no output.

\item LOG = 1  number of points read in, volume and energy range.

\item LOG = 2  like 1 plus a complete list of the points read in.

\end{itemize}

\end{description}

\item %
\begin{description}
\item[{\{newmol = 0\}: Enter newmol!= to create a new molecule or clean and restart}] \leavevmode 
an old molecule.

\end{description}

\end{itemize}

Required output variables:
%
\begin{itemize}

\item %
\begin{description}
\item[{molout: structure with the input molecular description. The format is:}] \leavevmode %
\begin{itemize}

\item molout.name -{}-{}-> title of the molecule.

\item %
\begin{description}
\item[{molout.atname -{}-> \{1:M\} cell array with the symbols of the atoms}] \leavevmode 
(M is the number of atoms in the molecule).

\end{description}

\item molout.xyz -{}-{}-{}-{}-> Mx3 matrix with the atomic coordinates.

\end{itemize}

\end{description}

\end{itemize}


\subsubsection{mol\_adduct.m%
  \label{mol-adduct-m}%
}


\subsubsection{mol\_align.m%
  \label{mol-align-m}%
}


\subsubsection{mol\_cmass.m%
  \label{mol-cmass-m}%
}


\subsubsection{mol\_dbatom.m%
  \label{mol-dbatom-m}%
}


\subsubsection{mol\_dbstart.m%
  \label{mol-dbstart-m}%
}


\subsubsection{mol\_dbsymbol.m%
  \label{mol-dbsymbol-m}%
}


\subsubsection{mol\_dist.m%
  \label{mol-dist-m}%
}


\subsubsection{mol\_dist2.m%
  \label{mol-dist2-m}%
}


\subsubsection{mol\_getfragment.m%
  \label{mol-getfragment-m}%
}


\subsubsection{mol\_inertiamatrix.m%
  \label{mol-inertiamatrix-m}%
}


\subsubsection{mol\_internalgeometry.m%
  \label{mol-internalgeometry-m}%
}


\subsubsection{mol\_islinear.m%
  \label{mol-islinear-m}%
}


\subsubsection{mol\_isnew.m%
  \label{mol-isnew-m}%
}


\subsubsection{mol\_isplanar.m%
  \label{mol-isplanar-m}%
}


\subsubsection{mol\_readxyz.m%
  \label{mol-readxyz-m}%
}


\subsubsection{mol\_transform.m%
  \label{mol-transform-m}%
}


\subsubsection{mol\_unitconvert.m%
  \label{mol-unitconvert-m}%
}


\subsubsection{mol\_uniqatoms.m%
  \label{mol-uniqatoms-m}%
}


\subsubsection{mol\_writeg09.m%
  \label{mol-writeg09-m}%
}


\subsubsection{mol\_writenw.m%
  \label{mol-writenw-m}%
}


\subsubsection{mol\_writexyz.m%
  \label{mol-writexyz-m}%
}


\subsubsection{op\_prod.m%
  \label{op-prod-m}%
}


\subsubsection{op\_rot3D.m%
  \label{op-rot3d-m}%
}


\subsubsection{op\_rotx.m, op\_roty.m, op\_rotz.m%
  \label{op-rotx-m-op-roty-m-op-rotz-m}%
}


\subsubsection{zmat\_step.m%
  \label{zmat-step-m}%
}


\subsection{1.7~~~Copyright notice%
  \label{copyright-notice}%
}

Copyright © 2011-2012, Víctor Luaña <\href{mailto:victor@carbono.quimica.uniovi.es}{victor@carbono.quimica.uniovi.es}>,
Departamento de Química Física y Analítica, Universidad de Oviedo,
E-33007 Oviedo, Principado de Asturias, Spain
and Alberto Otero-de-la-Roza <\href{mailto:aoterodelaroza@ucmerced.edu}{aoterodelaroza@ucmerced.edu}>, School of
Natural Sciences, University of California, Merced, 5200 North Lake Road,
Merced, California 95343, USA.

This program is free software; you can redistribute it and/or
modify it under the terms of the GNU General Public License
as published by the Free Software Foundation; either version 2
of the License, or (at your option) any later version.

This program is distributed in the hope that it will be useful,
but WITHOUT ANY WARRANTY; without even the implied warranty of
MERCHANTABILITY or FITNESS FOR A PARTICULAR PURPOSE.  See the
GNU General Public License for more details.

You should have received a copy of the GNU General Public License
along with this program; if not, write to the Free Software
Foundation, Inc., 59 Temple Place - Suite 330, Boston,
MA 02111-1307, USA.

\begin{thebibliography}{mw1}
\bibitem[mw1]{mw1}{
Víctor Luaña and Alberto Otero-de-la-Roza,
``MolWare: a package of molecular tools'',
\url{http://azufre.quimica.uniovi.es/software.html}
}
\end{thebibliography}

\end{document}
